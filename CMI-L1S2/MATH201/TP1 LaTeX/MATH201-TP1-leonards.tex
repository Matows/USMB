\documentclass[a4paper,12pt]{article}
\usepackage{amsfonts,amscd,,amssymb,amsmath, amsthm}
\usepackage{graphicx}
\usepackage[french]{babel}
\usepackage[utf8]{inputenc}
\usepackage[T1]{fontenc}
\usepackage{enumitem}
\usepackage{xcolor}

\setlength{\textwidth}{170mm} \setlength{\textheight}{260mm}
\setlength{\oddsidemargin}{-5mm} \setlength{\evensidemargin}{-5mm}
\setlength{\topmargin}{-10mm} \setlength{\headheight}{0mm}
\setlength{\headsep}{0mm} \setlength{\footskip}{10mm}
\setlength{\parindent}{0mm}

\newtheorem{exercic}{Exercice}
\def\exercice{\exercic\rm}

\def\dst{\triangleleft}

\def\R{\mathbb{R}}
\def\C{\mathbb{C}}
\def\N{\mathbb{N}}
\def\Z{\mathbb{Z}}
\def\Q{\mathbb{Q}}
\def\K{\mathbb{K}}

\newcommand{\add}[1]{\textcolor{blue}{#1}}


\begin{document}

{\LARGE\bf Universit\'e Savoie Mont Blanc\hfill{Ann\'ee 2019-2020}}

\smallskip

{\Large\bf UFR SceM\hfill L1 S2 MIST-CMI Informatique}

\smallskip

\begin{center}
{\bf
{\large MATH201-MIST}

\smallskip

{
TP alg\`ebre lin\'eaire
}
}
\end{center}

\medskip

\add{NB : \bsc{Dans les systèmes d'équations, les équivalences sont en faite des implications, je me suis trompé de symbole...}}\newline
{\bf Pr\'enom NOM} : \add{Simon \bsc{Léonard}}

\medskip

{\bf Groupe de TP/Nom} de l'intervenant : \add{CMI-INFO / Stéphane \bsc{Simon}}

\medskip

{\bf Date/horaire} de la s\'eance : \add{16-04-2020 / 9h - 12h}

\medskip


{\bf Consignes} : 
\begin{itemize}
    \item Se connecter \`a {\tt Cocalc.com}
    \item Cr\'eer un compte (avec l'adresse mel {\tt Prenom.Nom@etu.univ-smb.fr})
    \item Cr\'eer un nouveau projet (dossier) nomm\'e {\tt MATH201-MIST-2019-2020}
    \item Cr\'eer un nouveau fichier Jupiter Notebook nomm\'e {\tt MATH201-MIST-TP-Prenom-Nom.ipynb} dans ce projet
    \item S\'electionner le {\tt \emph{Kernel} SageMath (Development, Py3)}
    \item Répondre aux questions directement sur l'énoncé du TP et le rendre \emph{pr\'ecis\'ement} lors du 
    deuxi\` eme CC, 10 min. avant le d\'ebut de l'épreuve (absence au TP = 0/20, TP non rendu ce jour = 0/20)
\end{itemize}

\bigskip

\begin{exercice}[Matrices et produit matriciel]
    \begin{enumerate}
        \item Dans {\tt sage}, pour entrer la matrice
            $$M = \left(\begin{array}{ccc}
                1 & 2 & 3 \\
                4 & 5 & 6
            \end{array} \right)
            $$
            il suffit de taper {\tt M = matrix([[1,2,3],[4,5,6]])} et d'appuyer sur {\tt Ctrl-Entrée}.

            Pour voir si la matrice $M$ est bien saisie, taper M et appuyer sur {\tt Ctrl-Entrée}.

                 \item Dans {\tt sage} pour entrer le vecteur colonne 
            $$V= \left( \begin{array}{c} 7 \\ 8 \\ 9 \end{array} \right)$$
            taper {\tt V = matrix([[7],[8],[9]])} et appuyer sur {\tt Ctrl-Entrée}.

            Pour voir si le vecteur $V$ est bien saisi, taper {\tt V} et appuyer sur {\tt Ctrl-Entrée}.
        \item On considère un entier $n\geqslant 1$. On considère un vecteur ligne et un vecteur colonne à $n$ entrées
            $$
            L=(l_1,l_2,\dots,l_n)\:\:\:\:\text{et}\:\:\:\:
            C= \left(\begin{array}{c} c_1 \\ c_2 \\ \vdots \\ c_n \end{array}\right).$$
            On définit le produit $L\cdot C$ de la manière suivante 
            $$ L \cdot C = l_1c_1 + l_2c_2 + \cdots + l_n c_n = \sum_{i=1}^{n} l_i c_i.$$

            Effectuer à la main le produit $L\cdot C$ pour
            $$L=(1,2,3),\:C=\left(\begin{array}{c} -3 \\ 2 \\ -1 \end{array}\right),\:\: L\cdot C = \add{(-2)}$$
            Vérifier votre calcul en saisissant sous {\tt sage L} et {\tt C} et en effectuant le produit {\tt L*C}.


            On considère une matrice $M$ ayant $m$ lignes et $n$ colonnes et un vecteur colonne $C$
            $$ M = \left(\begin{array}{c} L_1 \\ L_2 \\ \vdots \\ \ L_m \end{array} \right),\\ 
            C=\left(\begin{array}{c} c_1 \\ c_2 \\ \vdots \\ c_n \end{array}\right).$$
            On définit le produit de $M$ par $C$ en effectuant le produit de chaque ligne de $M$ avec la colonne $C$ : 
            $$M\cdot C = \left(\begin{array}{c} L_1 \\ L_2 \\ \vdots \\ L_m \end{array} \right) \cdot
                               \left(\begin{array}{c} c_1 \\ c_2 \\ \vdots \\ c_n \end{array}\right)
                 = \left( \begin{array}{c} L_1\cdot C \\ L_2\cdot C \\ \vdots \\ L_m \cdot C \end{array} \right)$$
                 Effectuer à la main le produit 
            $$\left(
                    \begin{array}{ccc}
                    2 & 2 & 1 \\
                    0 & -1 & 1 \\
                \end{array}
               \right)\cdot
               \left(
                     \begin{array}{c}
                     1 \\ -1 \\ 1
                 \end{array}
               \right) = 
               \add{
                    \left(
                        \begin{array}{c}
                            1 \\ 2
                        \end{array}
                    \right)
               }
            $$
            Vérifier votre calcul avec {\tt sage} en entrant la matrice {\tt A} et le vecteur {\tt C} correspondant, puis en tapant {\tt A*C}.
        \item On définit le produit d'une matrice $T$ ayant $n$ colonnes avec une matrice $S$ ayant $n$ lignes en effectuant le produit
            de $T$ par les colonnes de $S$, successivement colonne par colonne :  
            $$T=\left(\begin{array}{c} L_1 \\ \vdots \\ L_i \\ \vdots \\ L_m \end{array} \right), \: 
            S=\left(\begin{array}{ccccc} C_1 & \cdots & C_j & \cdots & C_p \end{array} \right)$$
                  $$T.S = \left(
                                \begin{array}{ccccc}
                            L_1\cdot C_1 & \cdots & L_1\cdot C_j & \cdots & L_1\cdot C_p \\
                            \vdots &        & \vdots  &       & \vdots \\
                            L_i\cdot C_1 & \cdots & L_i\cdot C_j & \cdots & L_i\cdot C_p \\
                                                        \vdots &        & \vdots  &       & \vdots \\
                            L_m\cdot C_1 & \cdots & L_m\cdot C_j & \cdots & L_m\cdot C_p
                        \end{array}
                          \right).
                  $$
                  Observez que la matrice $T\cdot S$ a autant de lignes que $T$ et autant de colonnes que $S$.

                  Effectuer les produits suivants 
                  $$\left(
                          \begin{array}{ccc}
                          1 & 0 & -1 \\
                          1 & 1 & 0
                      \end{array}
                      \right) \cdot
                    \left(
                          \begin{array}{cc}
                          1 & 1 \\
                          1 & 0 \\
                          1 & 1
                      \end{array}
                      \right) = 
                      \add{
                          \left(
                            \begin{array}{cc}
                                0 & 0 \\
                                2 & 1
                            \end{array}
                          \right)
                      }
                  $$
                      et 
                $$\left(
                          \begin{array}{cc}
                          1 & 1 \\
                          1 & 0 \\
                          1 & 1
                      \end{array}
                      \right) \cdot
                      \left(
                          \begin{array}{ccc}
                          1 & 0 & -1 \\
                          1 & 1 & 0
                      \end{array}
                      \right) = 
                      \add{
                          \left(
                            \begin{array}{ccc}
                                2 & 1 & -1 \\
                                1 & 0 & -1 \\
                                2 & 1 & -1
                            \end{array}
                          \right)
                      }
                  $$
                  Vérifier vos calculs avec {\tt sage}.
          \end{enumerate}
\end{exercice}

\bigskip

\begin{exercice}[Matrices et applications linéaires]
    On considère l'application
    $$\begin{array}{ccccc}
        f & : & \R^3 & \to & \R^2 \\
          &   & (x,y,z) & \mapsto & (x-y,z-y).
      \end{array}
    $$
    \begin{enumerate} 
        \item Montrer que cette application est linéaire.
            \newline
            \add{
                Une application est linéaire si elle est additive et homogène.
                \begin{enumerate}
                    \item Soit $u=(x, y, z)$, $u'=(x', y', z') \in \R^3$, deux vecteurs.
                    \item Vérification de l'additivité :
                        \begin{align*}
                            f(u+u') &= f((x+x',y+y',z+z'))\\
                                    &= (x+x'-y-y', z+z'-y-y')\\
                                    &= (x-y+x'-y', z-y+z'-y')\\
                                    &= (x-y,z-y) + (x'-y',z'-y')\\
                                    &= f(u) + f(u')
                        \end{align*}
                    \item Vérification de l'homogénéité : 
                        \begin{align*}
                            f(\lambda u) &= f((\lambda x, \lambda y, \lambda z))\\
                                         &= (\lambda x-\lambda y,\lambda z-\lambda y)\\
                                         &= (\lambda (x-y), \lambda (z-y))\\
                                         &= \lambda (x-y, z-y)\\
                                         &= \lambda f(u)
                        \end{align*}
                \end{enumerate}
                Puisque l'application est additive et homogène, l'application est linéaire.
            }

        \bigskip
            
        \item On note $\mathcal C_3 = \left( (1,0,0), (0,1,0), (0,0,1) \right)$ et 
            $\mathcal C_2 = \left( (1,0), (0,1) \right)$ les bases canoniques de $\R^3$ et $\R^2$ respectivement.

            Montrer que les composantes des vecteurs $f(1,0,0)$, $f(0,1,0)$ et $f(0,0,1)$ dans la base $\mathcal C_2$ sont 
            $$
            [f(1,0,0)]_{\mathcal C_2} = \left( \begin{array}{c} 1 \\ 0 \end{array} \right),
            [f(0,1,0)]_{\mathcal C_2} = \left( \begin{array}{c} -1 \\ -1 \end{array} \right),
            [f(0,0,1)]_{\mathcal C_2} = \left( \begin{array}{c} 0 \\ 1 \end{array} \right).
            $$

            \add{
            Je vérifie donc pour chaque cas : 
                \begin{align}
                    f(1,0,0)&=(1-0,0-0)=(1,0)\\
                    f(0,1,0)&=(-1,-1)\\
                    f(0,0,1)&=(0,1)
                \end{align}
          }


        \item La matrice 
            $$mat(f,\mathcal C_3, \mathcal C_2)=\left( 
                  \begin{array}{c|c|c}
                      [f(1,0,0)]_{\mathcal C_2} & [f(0,1,0)]_{\mathcal C_2} & [f(0,0,1)]_{\mathcal C_2}
                  \end{array}             
                \right) 
               = 
                 \left( 
                   \begin{array}{ccc}
                       1 & -1 & 0 \\
                       0 & -1 & 1
                   \end{array}
                   \right)
                $$
            est appelé \emph{matrice} de $f$ dans les bases $\mathcal C_3$ et $\mathcal C_2$. 

        En utilisant le produit matriciel défini dans l'exercice 1, montrer que pour tout $x$, $y$, $z\in\R$ on a 
            $$[f(x,y,z)]_{\mathcal C_2} = mat(f,\mathcal C_3, \mathcal C_2)[(x,y,z)]_{\mathcal C_3}.$$
            \add{
                D'une part,
                \begin{align*}
                    [f(x,y,z)]_{\mathcal C_2}&=[f(x,0,0)]_{\mathcal C_2}
                         +[f(0,y,0)]_{\mathcal C_2}
                         +[f(0,0,z)]_{\mathcal C_2}
                         \\
                         &=x\cdot [f(1,0,0)]_{\mathcal C_2}
                         +y\cdot [f(0,1,0)]_{\mathcal C_2}
                         +z\cdot [f(0,0,1)]_{\mathcal C_2}
                         \\
                         &=x\cdot\left(\begin{array}{c} 1 \\ 0 \end{array}\right)
                         + y\cdot\left(\begin{array}{c} -1 \\ -1 \end{array}\right)
                         + z\cdot\left(\begin{array}{c} 0 \\ 1 \end{array}\right)
                         \\
                         &=\left(\begin{array}{c} x-y \\ -y+z \end{array}\right)
                \end{align*} 
                D'autre part,
                \begin{align*}
                    &\quad mat(f,\mathcal C_3, \mathcal C_2)\cdot[(x,y,z)]_{\mathcal C_3}\\
                    &=\left( 
                           \begin{array}{ccc}
                               1 & -1 & 0 \\
                               0 & -1 & 1
                           \end{array}
                     \right)
                     \cdot
                     \left(\begin{array}{c}x\\y\\z\end{array}\right)\\
                     &=\left(\begin{array}{c} x-y \\ -y+z \end{array}\right)
                \end{align*}
                Donc $[f(x,y,z)]_{\mathcal C_2} = mat(f,\mathcal C_3, \mathcal C_2)[(x,y,z)]_{\mathcal C_3}.$
            }
        \item Dans {\tt sage}, saisir dans la variable {\tt M} la matrice $mat(f,\mathcal C_3, \mathcal C_2)$ puis utiliser {\tt sage} pour calculer 
        
            $$
            [f(1234,5678,9101)]_{\mathcal C_2} = 
            \add{
                \left(\begin{array}{c} -4444\\ 3423 \end{array}\right)
            }
            $$
            
        \item Calculer le noyau de $f$. Justifier. Vérifier votre calcul avec {\tt sage} en tapant la commande {\tt M.right\_kernel()}

            \add{
            On cherche $ f(x,y,z)=0_{\R^2}=(0,0) $.
            \begin{align*}
                &\quad\quad\quad f(x,y,z)=(0,0) \\
                &\implies
                \left\{
                \begin{aligned}
                    x-y&=0\\
                    z-y&=0
                \end{aligned} \right.\\
                &\implies
                x=y=z\\
            \end{align*}
            donc $ker(f)=\{x=y=z\}=<(1,1,1)>$
            }

       \item Rappeler la définition du \emph{rang} de $f$. Le calculer directement à partir de la définition puis à l'aide du théorème du rang. Vérifier votre calcul en tapant dans {\tt sage} la commande {\tt M.image()}.

            \add{
                Le rang d'une application linéaire est la dimension du sous-espace vectoriel image de cette même application. Autrement écrit, $rg\,f = dim\,im\,f$.
                \begin{align*}
                    im\,f &= \{(a,b) \in \R^2 : \exists (x,y,z) \in \R^3, (a,b) = f(x,y,z)\}\\
                          &= \left\{(a,b) \in \R^2 : \exists (x,y,z) \in \R^3,
                           \left\{
                               \begin{aligned}
                                   x-y&=a\\
                                   z-y&=b
                               \end{aligned}
                           \right.
                           \quad
                          \right\}\\
                          &= \left\{(a,b) \in \R^2 : \exists (x,y,z) \in \R^3,
                           \left\{
                               \begin{aligned}
                                   x&=a+z-b\\
                                   y&=z-b
                               \end{aligned}
                           \right.
                           \quad
                          \right\}\\
                         &=\R^2
                \end{align*}
                Avec le théorème du rang :
                \begin{align*}
                    dim\,\R^3 &= dim(ker\,f) + rg(f)\\
                    rg(f)&=dim\,\R^3-dim(ker\,f)\\
                    &=3-1\\
                    &=2\\
                    &\implies im\,f=\R^2
                \end{align*}
                Donc $rg\,f=dim\,im\,f=dim\,\R^2=2$
            }

    \end{enumerate}
\end{exercice}

\bigskip

\begin{exercice}[Obtenir des résultats avec {\tt sage} et les vérifier]
    On considère l'application linéaire
    $$ \begin{array}{ccccc}
        f & : & \R^3 & \to & \R^3 \\
          &   & (x,y,z) & \mapsto & (x+y+z,x+y+z,x+y+z)
      \end{array}.$$
     \begin{enumerate}
         \item En utilisant {\tt sage}, donner une base du noyau de $f$.

            \add{
                Soit $M$ la matrice de l'application linéaire $f$, avec $M=\left(\begin{array}{ccc} 1 & 1 & 1\\1 & 1 & 1\\1 & 1 & 1\end{array}\right)$.
                \newline
                {\tt M.right\_kernel()} renvoie la matrice $\left(\begin{array}{ccc}1 & 0 & -1 \\ 0 & 1 & -1\end{array}\right)$. Donc $ker\,f=<(1,0,-1),(0,1,-1)>$.
            }
         \item Vérifier à la main que la famille obtenue est bien une base du noyau de $f$.

             \add{
                {\bf On vérifie que la famille est libre}\\
             Soit $\alpha,\beta \in \R$.
                 $$
                 \alpha(1,0,-1)+\beta(0,1,-1)=(0,0,0)
                 \iff
                 \left\{
                     \begin{aligned}
                         \alpha&=0\\
                         \beta&=0\\
                         -\alpha-\beta&=0
                     \end{aligned}
                 \right.
                 \iff
                 \alpha=\beta=0
                 $$
             Donc $<(1,0,-1),(0,1,-1)>$ est une famille libre.
            \newline
            \newline
             {\bf On vérifie que la famille est génératrice}\\
             Soit $(a,b,c)\in\R^3$.
             $$\alpha(1,0,-1)+\beta(0,1,-1)=(a,b,c)\iff
                 \left\{
                     \begin{aligned}
                         \alpha&=a\\
                         \beta&=b\\
                         -\alpha-\beta&=c
                     \end{aligned}
                 \right.
                \iff
                 \left\{
                     \begin{aligned}
                         \alpha&=a\\
                         \beta&=b\\
                         c&=-a-b
                     \end{aligned}
                 \right.
             $$
             Donc la famille est génératrice.
             }

     \end{enumerate}
    \end{exercice}
    

\begin{exercice}[Multiplication matricielle et composition d'applications].
On considère les applications linéaires
    
            $$\begin{array}{ccccc}
                f & : & \R^2 & \to & \R^3 \\
                  &   & (a,b) & \mapsto & (a+b, -a-b, a-b)
              \end{array},
              \:
              \begin{array}{ccccc}
                g & : & \R^3 & \to & \R^2 \\
                  &   & (x,y,z) & \mapsto & (x+z, x+y)
              \end{array}.
            $$
On note $\mathcal C_2$ et $\mathcal C_3$ les bases canoniques de $\R^2$ et $\R^3$ respectivement.
\begin{enumerate}
    \item Exprimer les composées 
        \begin{itemize}
            \item  $\begin{array}{ccccc}
                f\circ g & : & \R^3 & \to & \R^3 \\
                  &   & (x,y,z) & \mapsto & \add{
                    \begin{aligned}
                        f(x+z,x+y)&=(x+z+x+y,-x-z-x-y, x+z-x-y)\\
                        &=(2x+y+z,-2x-y-z,z-y)
                    \end{aligned}}
                    \end{array}$
            \item  $  \begin{array}{ccccc}
                g\circ f & : & \R^2 & \to & \R^2 \\
                  &   & (a,b) & \mapsto & \add{ g(a+b,-a-b,a-b)=(a+b+a-b,a+b-a-b)=(2a,0) }
                     \end{array}
                $
        \end{itemize}
    \item Calculer les matrices suivantes

        \begin{itemize}
            \item $M=mat(f,\mathcal C_2, \mathcal C_3) =\add{(\,f(1,0)|f(0,1)\,) = \left(\begin{array}{cc} 1 &1\\ -1 & -1 \\ 1 & -1\end{array}\right)}$

            \item $N=mat(g,\mathcal C_3, \mathcal C_2) = \add{(\,g(1,0,0)|g(0,1,0)|g(0,0,1)\,)=\left(\begin{array}{ccc}1 & 0 & 1 \\ 1 & 1 & 0\end{array}\right)}$

            \item $M\cdot N = \add{\left(\begin{array}{cc} 1 &1\\ -1 & -1 \\ 1 & -1\end{array}\right)\cdot\left(\begin{array}{ccc}1 & 0 & 1 \\ 1 & 1 & 0\end{array}\right)=\left(\begin{array}{ccc}2 & 1 & 1 \\ -2 & -1 & -1 \\ 0 & -1 & 1\end{array}\right)}$

            \item $N\cdot M = \add{\left(\begin{array}{ccc}1 & 0 & 1 \\ 1 & 1 & 0\end{array}\right)\cdot\left(\begin{array}{cc} 1 & 1\\ -1 & -1 \\ 1 & -1\end{array}\right)=\left(\begin{array}{cc} 2 & 0 \\ 0 & 0\end{array}\right)}$

            \item $A=mat(f \circ g,\mathcal C_3,\mathcal C_3)=\add{(\,f \circ g(1,0,0)|f \circ g(0,1,0)|f \circ g(0,0,1)\,)=\left(\begin{array}{ccc}2 & 1 & 1 \\ -2 & -1 & -1 \\ 0 & -1 & 1\end{array}\right)}$

                \item $B=mat(g\circ f,\mathcal C_2, \mathcal C_2)=\add{(\,g\circ f(1,0)|g\circ f(0,1)\,)=\left(\begin{array}{cc}2 & 0 \\ 0 & 0\end{array}\right)}$

        \end{itemize}
        Que constatez-vous ? 

        \add{Je constate que $A=M\cdot N$ et $B=N\cdot M$.}
    \item On considère l'application linéaire 
                   $$\begin{array}{ccccc}
                 h & : & \R^3 & \to & \R^3 \\
                  &   & (x,y,z) & \mapsto & (x+y+z, y+z, z)
                     \end{array}$$
           \begin{enumerate}
               \item Saisir dans {\tt sage} la matrice $$H = mat(h,\mathcal C_3,\mathcal C_3) =\add{\left(\begin{array}{ccc} 1 & 1 & 1 \\ 0 & 1 & 1 \\ 0 & 0 & 1\end{array}\right)}$$
               \item En utilisant {\tt sage}, calculer $H^k$ en choisissant divers entiers naturels $k$.
                   \begin{itemize}
                       \item Que conjecturez-vous pour $H^n$ avec $n\in \N$ ?

                            \add{
                                Je conjecture que $H^n=\left(\begin{array}{ccc}1 & n & \frac{n(n+1)}{2}\\ 0 & 1 & n\\ 0 & 0 & 1\end{array}\right)$
                           }
                       \item Vérifier votre conjecture avec {\tt sage} en calculant {\tt H$\mathtt{\wedge}$1000 = } \\

                           \add{
                               $H^{1000}=\left(\begin{array}{ccc}1 & 1000 & 500500 \\ 0 & 1 & 1000 \\ 0 & 0 & 1\end{array}\right)$
                           }
                       \item Prouvez votre conjecture.

                           \add{
                                Nous allons le démontrer par récurrence. Soit $P(n): H^n = \left(\begin{array}{ccc}1 & n & \frac{n(n+1)}{2}\\ 0 & 1 & n\\ 0 & 0 & 1\end{array}\right)$.
                                {\bf Initilisation}\\
                                $H^2=H\cdot H= \left(\begin{array}{ccc}1 & 2 & 3\\ 0 & 1 & 2 \\ 0 & 0 & 1\end{array}\right)=\left(\begin{array}{ccc}1 & 2=n & \frac{2(2+1)}{2}=\frac{n(n+1)}{2}\\ 0 & 1 & 2=n \\ 0 & 0 & 1\end{array}\right)$
                                \newline
                                {\bf Hérédité}\\
                                Soit $k\in\N$ et $H^k=\left(\begin{array}{ccc}1 & k & \frac{k(k+1)}{2}\\ 0 & 1 & k\\ 0 & 0 & 1\end{array}\right)$.
                                $$H^k\cdot H = \left(\begin{array}{ccc}1 & k & \frac{k(k+1)}{2}\\ 0 & 1 & k\\ 0 & 0 & 1\end{array}\right)\cdot \left(\begin{array}{ccc} 1 & 1 & 1 \\ 0 & 1 & 1 \\ 0 & 0 & 1\end{array}\right)=\left(\begin{array}{ccc}1 & k+1 & \frac{(k+1)(k+2)}{2}\\ 0 & 1 & k+1\\ 0 & 0 & 1\end{array}\right)=H^{k+1}$$
                                $P(n)$ et initialisé est héréditaire.
                            }
                   \end{itemize}
                       \item Expliciter la composée millième de $h$ avec lui-même
                                 $$\begin{array}{ccccc}
                                     h^{(1000)} & : & \R^3 & \to & \R^3 \\
                                                   &   & (x,y,z) & \mapsto & \add{(x+ny+z\frac{n(n+1)}{2},y+nz,z)}\\&&&& \add{=(x,1000y+\frac{n(n+1)}{2}z,z+1000z,z)}
                                   \end{array}$$
         \end{enumerate}
    \end{enumerate}
\end{exercice}

\begin{exercice}[Résolution d'un système linéaire avec {\tt sage} et applications]
    \begin{enumerate}
        \item Résoudre le système linéaire suivant en appliquant la méthode du pivot de Gauss
            $$\left\{
                \begin{aligned}
                    x+y+z &= 1 & L_1\\
                    x-y-z &= 2 & L_2\\
                    x-y+z &= 3 & L_3
                \end{aligned}
              \right.
              \add{
                  \iff
                  \left\{
                      \begin{aligned}
                        x+y+z&=1  &L_1\mapsto L_1\\
                        -2y-2z&=1 &L_2-L_1\mapsto L_2\\
                          -2y&=2  &L_3-L_1\mapsto L_3
                      \end{aligned}
                  \right.
                  \iff
                  \left\{
                      \begin{aligned}
                          x+y+z&=1 &L_1\\
                          -2y-2z&=1&L_2\\
                          y&=-1&L_3
                      \end{aligned}
                  \right.
              }
            $$
            $$
            \add{
            \iff
            \left\{
                \begin{aligned}
                      x+y+z&=1 &L_1\\
                      z&=\frac{1}{2}&L_2\\
                      y&=-1&L_3
                \end{aligned}
            \right.
            \iff
            \left\{
                \begin{aligned}
                    x&=\frac{3}{2} &L_1\\
                      y&=-1&L_3\\
                      z&=\frac{1}{2}&L_2
                \end{aligned}
            \right.
            }
            $$

        \item Ce système s'écrit aussi sous forme matricielle 
            $$\left(
                    \begin{array}{ccc}
                    1 & 1 & 1 \\
                    1 & -1 & -1 \\
                    1 & -1 & 1 
                \end{array}
              \right)
              \left(
                    \begin{array}{c}
                    x \\ y \\ z 
                \end{array}
              \right)
              =
              \left(
                    \begin{array}{c}
                    1 \\ 2 \\ 3 
                \end{array}
              \right).
              $$
              Saisir dans les variables {\tt S} et {\tt C} la matrice du système et le vecteur colonne du second membre.
              Pour résoudre le système avec {\tt sage} il suffit de saisir {\tt S.solve\_right(C)}. Vérifier vos calculs.

              \add{
                  {\tt S=matrix([[1,1,1],[1,-1,-1],[1,-1,1]])\newline C=matrix([[1],[2],[3]])\newline S.solve\_right(C)\newline\newline
                  OUTPUT:\newline
                  [3/2]\newline
                  [ -1]\newline
                  [1/2]
                  }
            }
          \item On considère l'application linéaire
              $$\begin{array}{ccccc}
                  f & : & \R^3 & \to & \R^2 \\
                    &   & (x,y,z) & \mapsto & (x+y+z,x-y-z)
                \end{array}
              $$
              \begin{enumerate}
                  \item Calculer une base du noyau de $f$. Vérifier vos calculs avec {\tt sage}.

                      \add{
                      D'abord, il nous faut résoudre le système $f(x,y,z)=(0,0)$.
                      $$
                      \left\{
                          \begin{aligned}
                              x+y+z&=0\\
                              x-y-z&=0
                          \end{aligned}
                      \right.
                      \iff
                      \left\{
                          \begin{aligned}
                              x+y&=-z&L_1\\
                              x-y&=z&L_2
                          \end{aligned}
                      \right.
                      \iff
                      \left\{
                          \begin{aligned}
                              x+y&=-z&L_1\\
                              -2y&=2z&L_2-L_1
                          \end{aligned}
                      \right.
                      \iff
                      \left\{
                          \begin{aligned}
                              x&=0\\
                              y&=-z
                          \end{aligned}
                      \right.
                      $$
                      Donc $ker\, f = \{(0,-z,z):z\in\R\}=\{z(0,-1,1):z\in\R\}=Vect((0,-1,1))$.
                      On a montré que cette famille est libre et sa dimension est la même que l'ensemble de départ, elle est donc aussi génératrice.
                      Donc la famille $\left((0,-1,1)\right)$ est une base de $kar\, f$.
                      }
                  \item Résoudre à la main le système $f(x,y,z)=(1,2)$.
                      \add{
                      $$
                      \left\{
                          \begin{aligned}
                              x+y+z&=1\\
                              x-y-z&=2
                          \end{aligned}
                      \right.
                      \iff
                      \left\{
                          \begin{aligned}
                              x+y&=1-z&L_1\\
                              x-y&=2+z&L_2
                          \end{aligned}
                      \right.
                      \iff
                      \left\{
                          \begin{aligned}
                              x+y&=1-z&L_1\\
                              2x&=3&L_1+L_2
                          \end{aligned}
                      \right.
                      $$
                      $$
                      \iff
                      \left\{
                          \begin{aligned}
                              y&=1-z-x\\
                              x&=\frac{3}{2}
                          \end{aligned}
                      \right.
                      \iff
                      \left\{
                          \begin{aligned}
                              y&=-\frac{1}{2}-z\\
                              x&=\frac{3}{2}
                          \end{aligned}
                      \right.
                      $$
                      $S=\{(\frac{3}{2},-\frac{1}{2}-z,z), z\in\R\}$
                      }
                  \item Montrer que la différence de deux solutions de ce système est un élément du noyau de $f$.

                      \add{
                      Prenons $S_1$ avec $z=1$ et $S_2$ avec $z=2$ : 
                      $$S_2-S_1=\left(\frac{3}{2}, -\frac{5}{2}, 2\right)-\left(\frac{3}{2}, -\frac{3}{2}, 1\right)=\left(\frac{3}{2}-\frac{3}{2},\frac{-5}{2}-\frac{-3}{2}, 2-1\right)=(0,-1,1)$$
                      Cette solution est bien un élément du noyau de $f$.
                      }
                  \item Utiliser {\tt sage} pour résoudre ce système. Que constatez-vous ? Comment feriez-vous en utilisant {\tt sage} pour obtenir toutes les solutions de ce système ? 

                      \add{
                          {\tt sage} me donne la solution pour $z=0$. Puisque $z\in\R$, il est impossible d'avoir l'ensemble des solutions. Mais si nous voulions un certain nombre de solutions, nous pourrions simplement exécuter une boucle qui fait varié $z$, avec un certain pas.
                      }
              \end{enumerate}
          \item On considère la famille de vecteurs 
              $$\vec{u}=(1,2,3), \vec{v}=(1,0,1), \vec{w}=(-1,6,5)$$
              Est-ce une famille libre ? une famille génératrice de $\R^3$ ? une base de $\R^3$ ? (vérifiez vos calculs avec {\tt sage})

              \add{
                  Soit $\alpha, \beta,\gamma \in\R$. Résolvons le système suivant : $\alpha \vec{u} + \beta \vec{v} + \gamma \vec{w} = \vec{0}$.
                  $$
                      \alpha(1,2,3)+\beta(1,0,1)+\gamma(-1,6,5)=(0,0,0)
                  $$
                  $$
                      \iff
                      \left\{
                      \begin{aligned}
                        \alpha + \beta -\gamma &=0 &L_1\\
                        2\alpha + 6\gamma &= 0 &L_2\\
                        3\alpha + \beta + 5\gamma &=0 & L_3
                      \end{aligned}
                      \right.
                      \iff
                      \left\{
                      \begin{aligned}
                        \alpha + \beta -\gamma &=0 &L_1\\
                        -2\beta+ 8\gamma &= 0 &L_2-2L_1\mapsto L_2\\
                        -2\beta + 8\gamma &=0 & L_3-3L_1\mapsto L_3
                      \end{aligned}
                      \right.
                      $$$$
                      \iff
                      \left\{
                      \begin{aligned}
                          \alpha + \beta -\gamma&=0&L_1\\
                          -2\beta+ 8\gamma &= 0&L_2\\
                          -2\beta + 8\gamma &=0&L_3-L_2\mapsto L_3
                      \end{aligned}
                      \right.
                  $$
                  La solution de ce système n'est pas unique, donc la famille $(\vec{u},\vec{v},\vec{w})$ n'est pas libre, elle n'est donc pas une base non plus.
                  %TODO: prouvé qu'elle n'est pas génératrice non plus
              }
          \item On considère les trois sous-espaces vectoriels suivants
              $$F=\{(x,y,z)\in \R^3 \mid x+y+z = 0\},\: G=\{(x,y,z)\in \R^3 \mid x-y+z=0\}$$
              et
              $$D=\{(t,t,t)\in \R^3 \mid t \in \R\}$$
             $F$ et $G$ sont-ils en somme directe ? $G$ et $D$ sont-ils en somme directe ? sont-ils supplémentaires ?
             Décomposer le vecteur $(1,2,3)$ comme somme d'un vecteur de $G$ et d'un vecteur de $D$. Cette décomposition est-elle unique ? 

             \add{
                 {\bf $F$ et $G$ en somme directe ?}\newline
                 $$
                 F\cap G:
                 \left\{
                 \begin{aligned}
                    x+y+z&=0\\
                    x-y+z&=0
                 \end{aligned}
                 \right.
                 \iff
                 \left\{
                 \begin{aligned}
                    y&=0\\
                    x&=-z
                 \end{aligned}
                 \right.
                 $$
                 Donc $S=\{(-z,0,z):z\in\R\}$. Puisque $F\cap G \ne \{0,0,0\}$, alors $F$ et $G$ ne sont pas en somme directe. $F$ et $G$ ne sont donc pas supplémentaires.
                 \newline
                 \newline
                 {\bf $G$ et $D$ en somme directe ?}\newline
                 $$
                 G\cap D:
                 \left\{
                 \begin{aligned}
                    x-y+z&=0\\
                     x=y&=z
                 \end{aligned}
                 \right.
                 \iff
                 \left\{
                 \begin{aligned}
                     y&=x+z\\
                     x&=y=z
                 \end{aligned}
                 \right.
                 $$
                 Donc $S=\{(0,0,0)\}$. Puisque $G\cap D = (0,0,0)$, alors $G$ et $D$ sont en somme directe. $dim\,D=1,\; dim\,G+dim\,D=dim(G\cap D) = 3$. $G$ et $D$ sont donc supplémentaires.
                 \newline
                 \newline
                 {\bf Décomposition de $(1,2,3)$ avec $G$ et $D$}\newline
                 On veut : $(x,y,z)_G + (t,t,t)_D = (1,2,3)_{\R^3}$. Donc :
                 $$
                 \left\{
                 \begin{aligned}
                     x-y+z&=0\\
                     x+t&=1\\
                     y+t&=2\\
                     z+t&=3
                 \end{aligned}
                 \right.
                 \iff
                 \left\{
                 \begin{aligned}
                     x&=-1\\
                     y&=0\\
                     z&=1\\
                     t&=2
                 \end{aligned}
                 \right.
                 $$
                 Donc $(1,2,3)=(-1,0,1)_G+(2,2,2)_D$. Par définition, la décomposition de (1,2,3) avec $G$ et $D$ est unique.
             }
         \item On considère une application linéaire $g:\R^n \to \R^m$.
             On considère une base $\mathcal E= (\vec e_1,\dots, \vec e_n)$ de $\R^n$ et une base 
             $\mathcal F = (\vec f_1,\dots,\vec f_m)$ de $\R^m$. La matrice de $g$ dans les bases $\mathcal E$ et $\mathcal F$ est définie par 
             $$mat(g,\mathcal E, \mathcal F)=\left([g(\vec e_1)]_{\mathcal F},\dots,[g(\vec e_n)]_{\mathcal F} \right).$$
             On considère l'application linéaire
             $$
             \begin{array}{ccccc}
                 g & : & \R^3 & \to & \R^2 \\
                   &   & (x,y,z) & \mapsto & (x+y+z,x-y+z)
               \end{array}
             $$
             On considère la base canonique de $\mathcal C_2$ de $\R^2$ et la base canonique $\mathcal C_3$ de $\R^3$.

             Soit les vecteurs 
             $$\vec u = (1,0,-1),\: \vec v = (1,2,0), \: \vec w=(0,1,2)$$
             et
             $$\vec a = (3,-1), \vec b = (3,0).$$
             \begin{enumerate}
                 \item Montrer que la famille $(\vec u, \vec v, \vec w)$ notée $\mathcal E$ est une base de $\R^3$. 
                     On pourra utiliser {\tt sage} pour les calculs.

                     \add{
                         Soit $ \alpha,\beta, \gamma \in \R$ avec $\alpha \vec{u}+\beta \vec{v}+\gamma \vec{w}=(0,0,0)$.
                         $$
                         \left\{
                         \begin{aligned}
                             \alpha + \beta &= 0\\
                             2 \beta + \gamma &= 0\\
                             - \alpha + 2 \gamma &= 0
                         \end{aligned}
                         \right.
                         \iff
                         \left\{
                         \begin{aligned}
                             \alpha &= 0\\
                             \beta&= 0\\
                             \gamma &= 0
                         \end{aligned}
                         \right.
                         $$
                     La famille est donc libre. De plus, $dim\, (\vec{u},\vec{v},\vec{w})=3$ donc la famille est une base de $\R^3$.
                     }
                 \item En utilisant {\tt sage} pour résoudre des systèmes linéaires, calculer les composantes suivantes 
                     $$[\vec u]_{\mathcal C_3},\: [\vec v]_{\mathcal C_3},\: [\vec w]_{\mathcal C_3},\:
                     [(1,0,0)]_{\mathcal E},\:[(0,1,0)]_{\mathcal E},\: [(0,0,1)]_{\mathcal E}$$ 

                     \add{
                         \setcounter{equation}{0}
                         \begin{align}
                             [\vec{u}]_{\mathcal C_3}&=(1,0,0)-0-(0,0,1)\\
                             [\vec{v}]_{\mathcal C_3}&=(1,0,0)+2(0,1,0)+0\\
                             [\vec{w}]_{\mathcal C_3}&=0+(0,1,0)+2(0,0,1)\\
                             [(1,0,0)]_{\mathcal E}&=\frac{4}{3} \vec{u} -\frac{1}{3} \vec{v} +\frac{2}{3} \vec{w}\\
                             [(0,1,0)]_{\mathcal E}&=-\frac{2}{3} \vec{u} +\frac{2}{3} \vec{v} -\frac{1}{3} \vec{w}\\
                             [(0,0,1)]_{\mathcal E}&=\frac{1}{3} \vec{u} -\frac{1}{3} \vec{v} +\frac{2}{3} \vec{w}
                         \end{align}
                     }
                 \item En déduire les matrices de l'identité $id_{\R^3} :\R^3 \to \R^3$\\
                     \begin{itemize}
                         \item $ mat(id_{\R^3},\mathcal C_3, \mathcal E) = \add{([(1,0,0)]_{\mathcal E}|[(0,1,0)]_{\mathcal E}|[(0,0,1)]_{\mathcal E})=\dfrac{1}{3}
                             \left(
                             \begin{array}{ccc}
                                 4 & -2 & 1\\
                                 -1 & 2 & -1\\
                                 2 & -1 & 2
                             \end{array}
                             \right)
                             }
                             $

                         \item $ mat(id_{\R^3},\mathcal E, \mathcal C_3) = \add{([\vec{u}]_{\mathcal C_3}|[\vec{v}]_{\mathcal C_3}|[\vec{w}]_{\mathcal C_3})=
                             \left(
                             \begin{array}{ccc}
                                 1 & 1 & 0\\
                                 0 & 2 & 1\\
                                 -1 & 0 & 2
                             \end{array}
                             \right)
                             }$

                     \end{itemize}
                     En utilisant sage, effectuez les produits suivants \\
                     \begin{itemize}
                         \item $ mat(id_{\R^3},\mathcal C_3, \mathcal E)\cdot mat(id_{\R^3},\mathcal E, \mathcal C_3) = \add{
                             \left(
                             \begin{array}{ccc}
                                 1 & 0 & 0\\
                                 0 & 1 & 0\\
                                 0 & 0 & 1
                             \end{array}
                             \right)}
                             $

                         \item $ mat(id_{\R^3},\mathcal C_3, \mathcal E)\cdot mat(id_{\R^3},\mathcal E, \mathcal C_3) = \add{
                             \left(
                             \begin{array}{ccc}
                                 1 & 0 & 0\\
                                 0 & 1 & 0\\
                                 0 & 0 & 1
                             \end{array}
                             \right)}$

                     \end{itemize}
                     A l'aide du lien entre produit matriciel et composée de fonctions, comment pouvait-on prévoir le résultat ?

                    \add{...}
                 \item Montrer que la famille $(\vec a, \vec b)$, notée $\mathcal F$, est une base de $\R^2$.

                     \add{
                         Soit $( \alpha , \beta ) \in \R^3, \alpha \vec{a} + \beta \vec{b} = \vec{0}$
                         $$
                         \left\{
                         \begin{aligned}
                             3 \alpha + 3 \beta = 0\\
                             - \alpha = 0\\
                         \end{aligned}
                         \right.
                         \iff
                         \left\{
                         \begin{aligned}
                             \alpha= 0\\
                             \beta= 0\\
                         \end{aligned}
                         \right.
                         $$
                         Donc $(\vec{a},\vec{b})$ est libre. $dim\, (\vec{a},\vec{b}) = 2$. Comme $(\vec{a},\vec{b})$ est libre et est de même dimension que $\R^2$, alors c'est aussi une base de $\R^2$.
                     }

                 \item En appliquant la méthode ci-dessus déterminer les matrices de l'identité 
                     $id_{\R^2} : \R^2 \to \R^2$ \\
                                         \begin{itemize}
                                             \item $ mat(id_{\R^2},\mathcal C_2, \mathcal F) = \add{([(1,0)]_{\mathcal F}|[(0,1)]_{\mathcal F})= \dfrac{1}{3}
                             \left(
                             \begin{array}{cc}
                                 0 & -3 \\
                                 1 & 3
                             \end{array}
                             \right)
                         } $ \\
                     
                     \item $ mat(id_{\R^2},\mathcal F, \mathcal C_2) = \add{([\vec{a}]_{\mathcal C_2}|[\vec{b}]_{\mathcal C_2})=
                             \left(
                             \begin{array}{cc}
                                 3 & 3 \\
                                 -1 & 0
                             \end{array}
                             \right)}$ \\
                 \end{itemize}

                 \item Donner les matrices
                     \begin{itemize}
                         \item $mat(g,C_3,C_2) = \add{([g(1,0,0)]_{\mathcal C_2}|[g(0,1,0)]_{\mathcal C_2}|[g(0,0,1)]_{\mathcal C_2})=
                             \left(
                             \begin{array}{ccc}
                                 1 & 1 & 1 \\
                                 1 & -1 & 1
                             \end{array}
                             \right)
                             }$
                         \item $mat(g,\mathcal E, \mathcal F) = \add{([g(1,0,-1)]_{\mathcal F}|[g(1,2,0)]_{\mathcal F}|[g(0,1,2)]_{\mathcal F})=
                             \left(
                             \begin{array}{ccc}
                                 0 & 1 & -1 \\
                                 0 & 0 & 2
                             \end{array}
                             \right)
                             }$
                     \end{itemize}
                 \item Avec {\tt sage} effectuer les calculs suivants \\
                     \begin{itemize}
                         \item $ mat(id_{\R^2},\mathcal C_2, \mathcal F)\cdot mat(g,C_3,C_2) \cdot mat(id_{\R^3},\mathcal E, \mathcal C_3) = \add{
                             \left(
                             \begin{array}{ccc}
                                 0 & 1 & -1 \\
                                 0 & 0 & 2
                             \end{array}
                             \right)
                             }$

                         \item $ mat(id_{\R^2},\mathcal F, \mathcal C_2) \cdot mat(g,\mathcal E, \mathcal F) \cdot mat(id_{\R^3},\mathcal C_3, \mathcal E)= \add{
                             \left(
                             \begin{array}{ccc}
                                 1 & 1 & 1 \\
                                 1 & -1 & 1
                             \end{array}
                             \right)
                             }$
                     \end{itemize}

                     Que remarquez-vous ? comment peut-on prévoir le résultat en faisant le lien entre le produit matriciel et la composition des applications linéaires ?

                     \add{
                     Je remarque que :
                     \begin{align*}
                         mat(g,C_3,C_2)&=mat(id_{\R^2},\mathcal F, \mathcal C_2) \cdot mat(g,\mathcal E, \mathcal F) \cdot mat(id_{\R^3},\mathcal C_3, \mathcal E)\\
                         mat(g,\mathcal E, \mathcal F)&=mat(id_{\R^2},\mathcal C_2, \mathcal F)\cdot mat(g,C_3,C_2) \cdot mat(id_{\R^3},\mathcal E, \mathcal C_3)
                     \end{align*}
                     }
             \end{enumerate}
    \end{enumerate}

\end{exercice}



\end{document}
 
