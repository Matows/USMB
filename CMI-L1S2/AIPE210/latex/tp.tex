\documentclass[11pt,a4paper]{article}

\usepackage[utf8]{inputenc} % ajouté
\usepackage[french]{babel}
\usepackage{fourier, erewhon}
\usepackage[T1]{fontenc}
%% permet de spécifier la langue 

\usepackage[margin=2cm]{geometry}
\usepackage{indentfirst}
\usepackage{verbatim}
\usepackage{enumerate}
%% pour inclure des figures
%%\usepackage{graphicx}
%% Pour les maths
\usepackage{amsthm}
\usepackage{amssymb}
\usepackage{amsmath}
\usepackage{color}
% \theoremstyle{plain}
% \newtheorem{thm}{Théorème}
% \newtheorem{prop}{Proposition}
% \newtheorem{exo}{Exercice}
% \theoremstyle{remark}
% \newtheorem{rem}{Remarque}

\newcommand{\grasitalique}[1]{\textit{\textbf{\color{blue} #1}}}
\newcommand{\diagonale}[3]{
\[
\begin{pmatrix}
	#1 & 0 & 0\\
	0 & #2 & 0\\
	0 & 0 & #3
\end{pmatrix}
\]
}

\parindent 12pt

\pagestyle{plain}


\title{TP {\LaTeX}}
\date{\today}
\author{Simon \bsc{Léonard}}

\begin{document}
\maketitle
\tableofcontents
\newpage

\section{Mises en formes}
	\begin{enumerate}
		\item[2.] \textbf{BOLD} and \textit{italic} and \textsc{something}
	\end{enumerate}

\section{Répondre aux challenges suivants}
	\begin{enumerate}
		\item On dérive la fonction \(f(x)=x^3\) et on trouve \(f'(x)=3x^2\)

		\item
		\[\sqrt{\alpha + 1} = \frac{1}{3}\]

		\item
		\[\Delta = a_{12} x^{12} + a_1 x + a_0\]

		\item
		\begin{equation} \label{eq1}
			f(x)= \cos{\left(\frac{\pi x}{2}\right)}
		\end{equation}

		\item
		\begin{equation}
			\int_0^1 -\frac{1-x}{1+x}dx
		\end{equation}

		\item
		C'est magnifique, merci.

		\item
		\[
		M=
		\begin{pmatrix}
		2 & 0 & 1\\
		0 & 1 & 2\\
		3 & 1 & 4 
		\end{pmatrix}
		\]

		\item
		\[\vec{u}_x=
		\begin{pmatrix}
			x\\	
			y\\
			z
		\end{pmatrix}
		\]

		\item
		Comme on le voit dans l'équation \ref{eq1}, blabla
	\end{enumerate}

\section{Commande personnalisée}
	Un mot en \grasitalique{gras et en italique coloré}.
	\begin{enumerate}
		\item \diagonale{1}{2}{8}
	\end{enumerate}

\end{document}
